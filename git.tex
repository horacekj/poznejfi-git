\documentclass[
]{beamer}

\usepackage[czech]{babel}
\usepackage[utf8]{inputenc}
\usepackage[T1]{fontenc}
\usepackage{csquotes}
\usepackage{expl3,biblatex}
\usepackage{hyperref}
\usepackage{booktabs}
\usetheme[
  workplace=fi,
]{MU}

\hypersetup{colorlinks,linkcolor=blue,urlcolor=blue}

\title{Verzovací systém git \& GitHub}
\subtitle{Workshop Poznej FI 2024}
\author[Honza Horáček]{Honza Horáček}
\institute[FI MU]{Spolek přátel severské zvěře, Fakulta informatiky, Masarykova univerzita}
\date{17. 2. 2024}
%\subject{}
%\keywords{}
\begin{document}

\begin{frame}[plain]
\maketitle
\end{frame}

\begin{frame}{Letem světem Linuxem}
\begin{enumerate}
	\item Login na tabuli.
	\item Budeme používat git v terminálu (wtf?!).
	\item Základy práce s terminálem: \\
	prompt, souborový systém, \texttt{ls}, \texttt{cd}, \texttt{Ctrl+C}.
	\pause
	\item Textové editory: \texttt{vim}, \texttt{nano}, ...
\end{enumerate}
\end{frame}

\begin{frame}
\includegraphics[width=\textwidth]{images/vim1.png}
\end{frame}

\begin{frame}
\centering
\includegraphics[height=\textheight]{images/vim2.jpg}
\end{frame}

\begin{frame}{Základní konfigurace gitu}

Ovládání: \\
\texttt{\$ git <command>} \\
\vspace{1em}

\texttt{\$ git config -{}-global user.name "Honza Horacek"} \\
\texttt{\$ git config -{}-global user.email "me@apophis.cz"} \\
\texttt{\$ git config -{}-global core.editor vim} :)

\begin{block}{Pro zvídavé}
\texttt{\$ git config -{}-list} \\
\texttt{\$ git config -{}-help}

Různé úrovně konfigurace: \texttt{-{}-system}, \texttt{-{}-global}, \texttt{-{}-local}
\end{block}
\end{frame}

\end{document}
